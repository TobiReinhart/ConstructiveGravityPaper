\documentclass{scrartcl}
\usepackage[utf8]{inputenc}

%bib
\usepackage[backend=biber, style=alphabetic, sorting=ynt]{biblatex}
\addbibresource{references.bib}
\usepackage{amsmath}
\usepackage{amssymb}
\usepackage[cal=boondoxo]{mathalfa}

%larger matrices
\setcounter{MaxMatrixCols}{20}

%commutative diagrams
\usepackage{tikz,tikz-3dplot}
\tdplotsetmaincoords{90}{45}
\tdplotsetrotatedcoords{-90}{180}{-90}


%drawing cones
\tikzset{surface1/.style={draw=black, fill=gray, fill opacity=.30}}
\newcommand{\coneback}[4][]{
  \draw[canvas is xy plane at z=#2, #1] (45-#4:#3) arc (45-#4:225+#4:#3) -- (O) --cycle;
  }
\newcommand{\conefront}[4][]{
  \draw[canvas is xy plane at z=#2, #1] (45-#4:#3) arc (45-#4:-135+#4:#3) -- (O) --cycle;
  }


%only linked url for websites
\AtEveryBibitem{%
  \ifentrytype{misc}{%
  }{%
    \clearfield{url}%
  }%
}

%citation nasa ads macros
\def\apjl{\textit{ApJ}}                % Astrophysical Journal, Letters
\def\apj{\textit{ApJ}}                % Astrophysical Journal, 
\def\aj{\textit{AJ}}                % Astronomical Journal, 
\def\aap{\textit{A\&A}}              % Astronomy & Astrophysics
\def\prd{\textit{Phys.~Rev.~D}}        % Physical Review D
\def\prl{\textit{Phys.~Rev.~Lett.}}    % Physical Review Letters
\def\jcap{\textit{J. Cosmology Astropart. Phys.}}
                % Journal of Cosmology and Astroparticle Physics
                

%theorem environment
\usepackage{amsthm}
\newtheorem{theorem}{Theorem}
\newtheorem{definition}{Definition}


%Links
\usepackage{hyperref}
\hypersetup{
    colorlinks,
    citecolor=black,
    filecolor=black,
    linkcolor=black,
    urlcolor=black
}

\title{Perturbative Approach to Covariant Constructive Gravity}
\author{Tobias Reinhart and Nils Alex }
\date{August 2019}

\begin{document}

\maketitle

\tableofcontents

\newpage

$\sim$ 20 pages

\section{Introduction}
1 page
\begin{itemize}
    \item principles of covariant constructive gravity
\end{itemize}

\section{The Axioms of Constructive Gravity}
\subsection{A1: Diffeomorphism Invariant Gravitational Dynamics }
\iffalse
2 pages
\begin{itemize}
    \item definition (tensor field bundles, jet bundles, equivariance, without intertwiners)
    \item equivariance equations
\end{itemize}
\fi

We wish to describe the gravitational field as a tensor field\footnote{The following developments can readily be generalized to the case of describing the gravitational field as a section of any bundle that is associated to the frame bundle, i.e., any natural bundle (cf. \cite{kolar1993natural}).} over the $4$-dimensional spacetime manifold $M$. To that end let $F \subset T^m_nM$ be a vector subbundle of the $(m,n)$ tensor bundle over $M$, such that the gravitational field can be described as a section $G \in \Gamma(F)$ of this bundle, the \textit{\textbf{gravitational field bundle}}. 
We denote adapted coordinates\footnote{As $F$ is a vectorbundle we can and will always restrict to coordinates that are linear on the fibers (cf. \cite{saunders_1989}).} on $F$ by $(x^m,v_A)$, where we introduced the abstract index $A$ that consequently runs over the fiber dimension of $F$.

As $F$ defines a vector bundle we can define its vector bundle dual $F^{\ast}$ with fiber at $p\in M$ given by the vector space dual of $\pi_F^{-1}(p)$.
Moreover, we denote fiber coordinates dual to $v_A$ by $v^A$, i.e, these two sets of coordinate functions satisfy:
\begin{align}
    v^Av_B = \delta^A_B.
\end{align}

We might very well consider the case where $F$ represents a true subbundle of $T^m_nM$ and hence admits fibers of dimension $r < m+n$. For such situations it is convenient to introduce vector bundle morphisms that relate fiber coordinates $v_A$ on $F$ to fiber coordinates $v^{a_1 ... a_m}_{b_1 ... b_n}$ on $T^m_nM$.

\begin{definition}[intertwiner]\label{interDef}
Let $(F,\pi_F,M)$ be a vector bundle. We call a pair of vector bundle morphisms $(I, J)$:
\begin{align}
    \begin{aligned}
    I&: F \longrightarrow T^m_n M\\
    J&: T^m_n M \longrightarrow F 
    \end{aligned}
\end{align}
that cover $id_M$ and satisfy
$J \circ I = \mathrm{id}_F$ a pair of \textbf{\textit{intertwiners}} for the bundle $(F, \pi_F, M)$.
\end{definition}

These intertwiners relate fiber coordinates on $F$ to fiber coordinates on $T^m_nM$

\begin{align} \label{interRel1}
    \begin{aligned}
    & v^{a_1 ... a_m}_{b_1 ... b_n} & = & \ \ I^{A a_1 ... a_m}_{b_1 ... b_n} \cdot v_{A},\\  
    & v_A & = & \ \ J^{b_1 ... b_n}_{A a_1 ... a_m} \cdot v^{a_1 ... a_m}_{b_1 ... b_n},
    \end{aligned}
\end{align}

as well as fiber coordinates on $F^\ast$ to fiber coordinates on $T^n_mM$

\begin{align} \label{interRel2}
    \begin{aligned}
    & v^{b_1 ... b_n}_{a_1 ... a_m} & = & \ \ J^{b_1 ... b_n}_{A a_1 ... a_m} \cdot v^{A},\\  
    & v^A & = & \ \  I^{A a_1 ... a_m}_{b_1 ... b_n} \cdot v^{b_1 ... b_n}_{a_1 ... a_m}.
    \end{aligned}
\end{align}

The identity $J\circ I = \mathrm{id}_F$ reads in this coordinate representation
\begin{equation}
    \delta^A _ B \ \ = \ \ I^{A a_1 ... a_m}_{b_1 ... b_n} \cdot J^{b_1 ... b_n}_{B a_1 ... a_m}.  
\end{equation}

The dynamics of the gravitational field shall be encoded as equations of motion to a second-derivative-order \textit{\textbf{Lagrangian}} for the gravitational field. We deliberately restrict to second-derivative-order Lagrangians as any higher-derivative-order contribution necessarily also contributes to the equations of motion in higher than second derivative order. These contributions would then, however, yield to instabilities in the associated Hamiltonian formulation (cf. \cite{Ostrogradsky:1850fid}, \cite{2015arXiv150602210W}). 

Such a gravitational Lagrangian can be rigorously defined by utilizing the \textit{\textbf{jet bundle}} framework\footnote{Much information regarding the jet bundle construction is also provided in \cite{saunders_1989}, \cite{seiler1994analysis}, \cite{seiler2009involution} and also in \cite{kolar1993natural}. } (cf. \cite{Gotay1992StressEnergyMomentumTA}, \cite{1998physics...1019G}). 
We denote adapted coordinates of the second order jet bundle $J^2F$ over $F$ by $(x^m, v_A, v_{Ap}, v_{AI})$. Here we introduced a new type of abstract indices that is used to label second order spacetime derivatives and thus, using that partial derivatives commute, runs from $0$ to $9$. 
The relation to the spacetime derivatives in standard notation is provided by an additional pair of intertwiners for the symmetric bundle $S_2M\subset T^0_2M$:
\begin{align}
    \begin{aligned}
        v_{AI} &= J_I^{ij} v_{Aij}\\
    v_{Aij} &= I^I_{ij} v_{AI}.
    \end{aligned}
\end{align}
We can now define a second-derivative-order gravitational Lagrangian as follows.
\begin{definition}[Lagrangian]
A second-order Lagrangian on $(F,\pi_F,M)$ is a bundle map that covers $id_M$:
\begin{align}
    \mathcal{L} : J^2F \longrightarrow \Lambda^4M.
\end{align}
\end{definition}
Thus the formulation of classical Lagrangian field theory yields the following situation, that can be seen in figure \ref{diagram1}:
\begin{figure}[hbt!]
\centering
\begin{tikzpicture}
\node (M) at (0,0) {$M$};
\node (F) at (0,2) {$F$};
\node (J1) at (0,4) {$J^1F$};
\node (J2) at (0,6) {$J^2F$};
\node (Vol) at (6,6) {$\Lambda^4M$};
\draw [-latex] (F) -- node[pos=0.35, right] {$\pi_F$} (M);
\draw [-latex] (J1) -- node[pos=0.35, right] {$\pi_{1,0}$}  (F);
\draw [-latex] (J2) -- node[pos=0.35, right] {$\pi_{2,1}$}  (J1);
\draw [-latex] (Vol.220) -- node[pos=0.4, left] {$\pi_{\Lambda^4M}$ \ }  (M.60);
\draw[-latex] (M) .. controls (-0.75,0.5) and (-0.75,1.5) .. node[pos=0.5, left] {$G$} (F);
\draw[-latex] (M) .. controls (-3,1) and (-3,5) .. node[pos=0.5, left] {$j^2(G)$} (J2);
\draw [-latex] (J2) -- node[pos=0.5, above] {$\mathcal{L}$}  (Vol);
\draw[-latex] (M.30) -- node[pos=0.5, right] { \ $\mathcal{L}\circ j^2(G)$} (Vol.250);
\end{tikzpicture}
\caption{Commutative Diagram: Lagrangian Field Theory on $J^2F$.} \label{diagram1}
\end{figure}
The gravitational field is described as a section of a bundle $F$ over the spacetime manifold $M$. As such it can be prolonged to any jet bundle $J^qF$, constructed over $F$, by applying the \textit{\textbf{jet prolongation}} map $j^q$. 
The Lagrangian $\mathcal{L}$ is a volume-form-valued bundle map on $J^2F$. Therefore we can prolong a given field $G \in \Gamma(F)$ to $J^2F$ and apply $\mathcal{L}$ on it to obtain a volume form on $M$, which then can be integrated.
This defines the usual \textit{\textbf{local action functional}} on the space of fields:
\begin{align}
\begin{aligned}
    \mathcal{S}_{\mathcal{L}} : \Gamma(F) &\longrightarrow \mathcal{R} \\
    G &\longmapsto \mathcal{S}_{\mathcal{L}}[G] := \int \mathcal{L}(j^2(G)).
\end{aligned}
\end{align}
Equations of motion (EOM) can be obtained by equating the \textit{\textbf{variational derivative}} of the Lagrangian with zero:
\begin{align}
  0 = E^A = \frac{\delta \mathcal{L}}{\delta v_A} = 
  \frac{\partial\mathcal{L}}{\partial v_A} - D_p ( \frac{\partial \mathcal{L}}{\partial v_{Ap}}) 
  + D_p D_q J^{pq}_I (\frac{\partial \mathcal{L}}{\partial v_{AI}}).
\end{align}
Here we further introduced the jet bundle \textit{\textbf{total derivative}}, sometimes also called formal derivative, $D_p$.
Given a function $f$ on $J^1F$, applying $D_p$ yields a function on the second order jet bundle:
\begin{align}
    D_p f := \frac{\partial f}{\partial x^p} + v_{Ap} \cdot  \frac{\partial f}{\partial v_A} + v_{AI} I^{I}_{pq} \cdot \frac{\partial f}{ \partial v_{Aq}}.
\end{align}
Note in particular that the EOM of a second-order Lagrangian are in general given by a function on $J^4F$, as we wish to restrict to theories that allow for a meaningful Hamiltonian formulation, we will restrict, however, to those cases where $\mathcal{L}$ is \textit{\textbf{degenerate}}, s.t. the EOM are also of second derivative order. 


One of the fundamental requirements that we wish to pose on the yet to be constructed gravitational dynamics is their \textit{\textbf{invariance under spacetime diffeomorphisms}}. 
This can be understood as a consequence of Einstein's requirement of general covariance (cf. \cite{Stachel1993-STATMO-5}, \cite{Pooley} and also \cite{Norton1993-NORGCA}).
To that end it is necessary that we lift the standard action of $\mathrm{Diff}(M)$ to $J^2F$.
As $F$ was required to be a subbundle of a tensor bundle over $M$, the action of $\mathrm{Diff}(M)$ lifts naturally by the usual pushforward-pullback construction to an action of $\mathrm{Diff}(M)$ on $F$ by vector bundle isomorphisms.
In the following we denote the image of $\phi \in \mathrm{Diff}(M)$ under this lift by $\phi_F$.
In order to further lift this action to the jet bundle we need to introduce some additional techniques from the theory of jet bundles.
\begin{definition}[prolongation of morphisms]
Let $(F_1,\pi_{F_1},M)$ and $(F_2,\pi_{F_2},N)$ be bundles, $\phi : M \rightarrow N$ a diffeomorphism, $f : F_1 \rightarrow F_2$ a bundle morphism covering $\phi$. The $k$th-order jet bundle lift of $(f,\phi)$ is the unique  map $j^k(f):J^kF_1 \rightarrow J^kF_2$ that lets the diagram in figure \ref{ProlongMorph} commute.
\begin{figure}[hbt!]
\centering
\begin{tikzpicture}
\node (M) at (0,0) {$M$};
\node (N) at (5,0) {$N$};
\node (F1) at (0,3) {$F_1$};
\node (F2) at (5,3) {$F_2$};
\node (JF1) at (0,6) {$J^kF_1$};
\node (JF2) at (5,6) {$J^kF_2$};
\draw [-latex] (M.10) -- node[pos=0.5, above] {$\phi$} (N.170);
\draw [<-] (M.350) -- node[pos=0.5, below] {$\phi^{-1}$} (N.190);
\draw [-latex] (F1) -- node[pos=0.5, above] {$f$} (F2);
\draw [-latex] (F1) -- node[pos=0.5, left] {$\pi_{F_1}$} (M);
\draw [-latex] (F2) -- node[pos=0.5, right] {$\pi_{F_2}$} (N);
\draw [-latex] (JF1) -- node[pos=0.5, left] {$(\pi_1)_{1,0}$} (F1);
\draw [-latex] (JF2) -- node[pos=0.5, right] {$(\pi_2)_{1,0}$} (F2);
\draw [-latex] (JF1) -- node[pos=0.5, above] {$j^k(f)$} (JF2);
\end{tikzpicture}
\caption{Commutative Diagram: Prolongation of Bundle Morphisms to First Jet Bundle.}\label{ProlongMorph}
\end{figure}
\end{definition}
Note that when acting on sections $G \in \Gamma(F_1)$, the jet bundle lift of bundle morphisms commutes with the jet prolongation map:
\begin{align}
j^k(f) \circ j^kG \circ \phi^{-1} = j^k \left (
f \circ G \circ \phi^{-1} \right ).
\end{align}

By using this notion of lifting bundle morhisms to the jet bundle we can finally formulate the first fundamental requirement of constructive gravity in rigorous fashion. 
\begin{definition}
A Lagrangian field theory described by a second-order Lagrangian $\mathcal{L} : J^2F \rightarrow \Lambda^4 M$ is called diffeomorphism invariant if $\mathcal{L}$ is equivariant w.r.t. the lifted action of $\mathrm{Diff}(M)$ on $J^2F$ and the pullback action on $\Lambda^4M$, i.e., if it holds for all $\phi \in \mathrm{Diff}(M)$ that 
\begin{align}\label{DiffeoReq}\tag{A1}
     \mathcal{L}\circ j^2(\phi_F) = \phi_{\ast} \circ \mathcal{L}.
\end{align}
\end{definition}

Infinitesimally, on the \textit{\textbf{Lie algebra level}}, diffeomorphisms are described by vector fields $\Gamma(TM)$ with lie bracket provided by their commutator. As usual we can obtain a Lie algebra ation from a given action of the corresponding Lie group (cf. \cite{boothby1989}, \cite{doi:10.1142/3867}).
This construction can in particular be used for the lifted action of $\mathrm{Diff}(M)$ on $F$ and on $J^2F$.
Doing so, we obtain a Lie algebra morphism:
\begin{align}\label{LieF}
\begin{aligned}
    \mathcal{f} : \Gamma(TM) &\longrightarrow \Gamma(TF)\\
    \xi &\longmapsto \xi_F \\
    \smallskip
    \xi_F = \xi^m \frac{\partial}{\partial x^m} + \xi^A \frac{\partial}{\partial v_A} &= \xi^m \frac{\partial}{\partial x^m} + C_{An}^{Bm} v_B \partial_m \xi ^n \frac{\partial}{\partial v_A}. 
\end{aligned}
\end{align}
Here we introduced the constant tensors $C^{Am}_{Bn}$ that describe the vertical coefficient of this lifted vector field.
Further, we get the following Lie algebra morphism that describes the corresponding vector field on $J^2F$:
\begin{align}
    \begin{aligned}
    j^2(\mathcal{f}) : \Gamma(TM) &\longrightarrow \Gamma(TJ^2F)\\
    \xi & \longmapsto \xi_{J^2F},
    \end{aligned}
\end{align}
where 
\begin{align}\label{LieJ2}
\begin{aligned}
    \xi_{J^2F} = &\hphantom{-} \xi^m \frac{\partial}{\partial x^m} + C_{An}^{Bm} v_B \partial_m \xi ^n \frac{\partial}{\partial v_A}
    + C_{An}^{Bm} \partial_m \xi^n v_{Bi} \frac{\partial}{\partial v_{Ai}}\\
    &- v_{An} \partial_m \xi ^n \frac{\partial}{\partial v_{Am}} + C_{An}^{Bm} v_B \partial_m \partial_p \xi^n \frac{\partial}{\partial v_{Ap}} 
    + C_{An}^{Bm} v_{BI} \partial_m \xi ^n \frac{\partial}{\partial v_{AI}}\\
    &- 2 v_{BJ} I^J_{an}J^{am}_I \partial_m \xi^n \frac{\partial}{\partial v_{AI}} + 2 C_{An}^{Bm} v_{Ba}J^{ap}_I \partial_m \partial_p \xi^n \frac{\partial}{\partial v_{AI}}\\
    &- v_{An} J^{pm}_I \partial_m \partial_p \xi^n\frac{\partial}{\partial v_{AI}} + C_{An}^{Bm} v_B J^{pq}_I \partial_m \partial_p \partial_q \xi^n \frac{\partial}{\partial v_{AI}}.
\end{aligned}
\end{align}
 
We can use the Lie algebra morphism (\ref{LieJ2}) to derive an infinitesimal version of the first fundamental requirement (\ref{DiffeoReq}).
\begin{theorem}
Let $\mathcal{L} = L \cdot \mathrm{d}^4x$ be the Lagrangian of a diffeormorphism invariant field theory on $J^2F$, i.e., $\mathcal{L}$ is assumed to satisfy condition (\ref{DiffeoReq}). Then the coordinate expression $L$ necessarily satisfies the following first-order, linear partial differential equation:  
\begin{align}\label{DiffeoEqn}
\begin{aligned}
    0 &= L^{:m} \\
    0 &= L^{:A} C_{An}^{Bm} v_B + L^{:Ap} \bigl[ C_{An}^{Bm} \delta_p^q - \delta_A^B \delta_m^n \bigr] v_{Bq} + L^{:AI} \bigl[ C_{An}^{Bm} \delta_I^J - 2 \delta_A^B J_I^{pm} I^J_{pn}  \bigr] v_{BJ} + L \delta^m_n \\
    0 &= L^{:A(p\vert}C_{An}^{B \vert m)} v_B + L^{: AI} \bigl[ C_{An}^{B(m\vert} 2 J_I^{\vert p) q} - \delta^B_A J_I ^{pm} \delta_n^q \bigr] v_{Bq} \\
    0 &= L^{:AI} C_{An}^{B(m\vert} v_B J_I^{\vert p q )}.
\end{aligned}
\end{align}
\end{theorem}
\begin{proof}
Reexpressing condition (\ref{DiffeoReq}) infinitesimally, by utilizing the Lie algebra morphism (\ref{LieJ2}) for an arbitrary vector field $\xi \in \Gamma(TM)$, yields an equation with left-hand side given by applying $\xi_{J^2F}$ on $L$ and right-hand side given by the infinitesimal of the pullback action of $\phi$ on $\Lambda^4M$.
As $\xi \in \Gamma(TM)$ was assumed arbitrary we can chose the individual components s.t. we can isolate specific contributions to the equation. These then have to be satisfied independently. Several suitable choices for the vector field components then yield precisely the above PDE. 
\end{proof}

This system of $140$ first-order, linear partial differential equations for the Lagrangian follows necessarily from the requirement of diffeomorphism invariance. Therefore the notoriously difficult requirement of diffeomorphism invariant gravitational dynamics is translated into the much simpler condition that the gravitational Lagrangian be a solution to (\ref{DiffeoEqn}). Conversely, every solution to this PDE yields a valid candidate Lagrangian to describe the gravitational dynamics. 
The problem of constructing gravitational dynamics is thus rephrased as computing the general solution to (\ref{DiffeoEqn}). 
This is an enormous advantage. As solving partial differential equations is a frequently occurring problem in almost all areas of research, the underlying theory is extensively developed (cf. \cite{seiler2009involution}, \cite{hormander1994analysis}, \cite{hormander2009analysis}, \cite{hormander2015analysis}).
Moreover, many techniques how such equations might be solved have already been known for quite some time (see \cite{Hilbert}).

Furthermore, note that the only quantity appearing in (\ref{DiffeoEqn}) that explicitly depends on the specific nature of the gravitational field is the vertical coefficient of the Lie algebra action of diffeomorphisms on $M$. This allows for a unified treatment of the PDE, irrespective of the precise gravitational field at hand. 

Similar equations were already obtained in the context of variational calculus and gauge symmetries (cf. \cite{article}). 
Gotay et al. derived a similar system for the case of first-order Lagrangians (see \cite{Gotay1992StressEnergyMomentumTA} and \cite{1998physics...1019G}) and used it to define a universal, conserved energy-momentum-tensor as Noether current associated to $\mathrm{Diff}(M)$. 
Unfortunately, restricting to first-order Lagrangians for a description of gravity does not always suffice --- for instance it does not when formulating GR such that the usual metric tensor constitutes the only dynamical field.

There are many exciting implications of this diffeomorphism equivariance PDE that are , however, beyond the scope of this paper. We have already presented a framework of constructing perturbative gravitational dynamics that thrives on  consequences of (\ref{DiffeoEqn}) on the corresponding EOM (cf. \cite{TobiR}).
Utilizing Gotay et al.'s jet bundle based formulation of Hamiltonian dynamics (see \cite{2004math.ph..11032G}) we have shown --- at least for first-derivative-order theories --- that the Hamiltonian associated to any diffeomorphism invariant Lagrangian field theory is \textit{\textbf{fully constrained}} and furthermore the Poisson algebra that is generated by the constraints necessarily represents the underlying diffeomorphism group in terms of \textbf{\textit{hypersurface deformations}} (for these results see \cite{TobiMaster}, in particular chapters 2.3 and 2.4).

Finally, with these results at hand, it is worth taking a closer look at three different ways how the gravitational dynamics that govern Einstein's General relativity were historically recovered as unique, second-derivative-order metric theory of gravity from first principles.
First of all, Loverlock (see \cite{Lovelock1969}, \cite{doi:10.1063/1.1665613} and \cite{doi:10.1063/1.1666069}) directly imposed the condition of diffeomorphism invariant EOM on the gravitational dynamics and used this to sucessfully recover the Einstein tensor.
Hojman et al. derived the canonical formulation of General Relativity by reacquiring the constraint algebra to resemble the algebra of hypersurface deformations.
And last but not least, contributions, mainly due to Deser, revealed how Einstein dynamics can be obtained by posing the condition of energy-momentum-conversation on the gravitational dynamics (cf. \cite{1970GReGr...1....9D}). 

The connection of Lovelock's fundamental requirement to our first condition (\ref{DiffeoReq}) is obvious --- after all, although restricted to metric theories and on level of the equations of motion, he directly required diffeomorphism invariance of the gravitational dynamics. 
Employing our results, we, however, now see that also Hojman et al. and Deser, maybe unknowingly, used diffeomorphism invariance as fundamental guiding principal. 
Hojman et al. required the canonical constraint algebra to represent the algebra of hypersurface deformations; as we have shown in \cite{TobiMaster} this is a necessary consequence of the PDE (\ref{DiffeoEqn}). 
Deser imposed energy-momentum conversation; Gotay et al. showed in \cite{Gotay1992StressEnergyMomentumTA} that their universal energy momentum tensor is conserved and moreover, reproduces the well-known expression in the case of General Relativity.
concluding, also these two methods of reproducing the GR dynamics from first principles ultimately impose nothing but diffeormorphism invariance of the gravitational dynamics. 

%comment on Schuller et al. ?




\subsection{A2: Causal Compatibility between Matter and Gravity}
\iffalse
3 pages 
\begin{itemize}
    \item principal polynomial (principal symbol, wkb, Itin technique)
    \item causal compatibility
    \item general implications
\end{itemize}
\fi 
In the last section we considered the formulation of a bare gravitational theory. If we additionally endow spacetime with a matter field $\phi \in  \Gamma(F_{mat})$ that is coupled to the gravitational field, i.e. whose dynamics is governed by a first-order Lagrangian
\begin{align}
    \mathcal{L}_{mat} : F_\text{grav} \times J^1F_\text{mat} \longrightarrow \Lambda^4M,
\end{align}
we additionally have to ensure that the description of matter and gravity are \textit{\textbf{causally compatible}}.

The causal structure of a given second-order EOM $E^A=0$ is closely related to the behaviour of wave-like solutions in the infinite frequency limit (cf. \cite{2018PhRvD..97h4036D}), i.e. the limit of geometrical optics. In order to take this limit, we consider the WKB ansatz for the coordinate expression of a section $G_A \in \Gamma(F)$
\begin{align}\label{waveAns}
    G_A(x^m) = \mathrm{Re}\left \{ e^{\frac{iS(x^m)}{\lambda}} \cdot   \bigl [ a_A(x^m) + \mathcal{O}(\lambda) \bigr ]\right \}.
\end{align}
Plugging this into the EOM and taking the limit $\lambda \rightarrow 0$ on obtains in leading order:
\begin{align}
    \underbrace{\left ( \frac{\partial E^A }{\partial v_{BI}} \right ) J_{I}^{ab} k_a k_b}_{T^{AB}(k_a)} a_B(x^m) = 0,
\end{align}
where now $k_a = - \partial_aS(x^m)$ is the wave covector of the ansatz. The $r\times r$ matrix $T^{AB}(k_a)$ is called the \textit{\textbf{principal symbol}} of the EOM. If the wave ansatz (\ref{waveAns}) with wave covector $k_a$ shall provide a non-trivial solution with $a_A \neq 0$ to the EOM, then, in particular,  $T^{AB}(k_a)$ must be non-injective. 

Requiring such a square matrix to be non-injective is of course equivalent to imposing the condition that its determinant be zero. There is, however, a caveat that obstructs this straight forward approach. If the theory at hand features gauge symmetries, its principal symbol is necessarily non-injective, irrespective of the specific covector $k_a$. 
The reason for this lies in the fact that for a gauge symmetry with $s$-dimensional orbits, there exist $s$ independent coefficient functions $\chi_{(i)A}(k_a)$, for $i = 1,...,s$, that are gauge-equivalent to the trivial solution $a_A(x^m)=0$ and thus are contained in the kernel of the principal symbol matrix. 
Consequently, if we wish to obtain at least one physically non-trivial solution with wave covector $k_a$ that does not vanish modulo gauge transformation we need to require that the kernel of $T^{AB}(k^m)$ is at least $s+1$ dimensional. This is equivalent to imposing the vanishing of all order-$s$ sub determinants, i.e., a vanishing order-$s$ adjunct matrix
\begin{align}
    Q_{(A_1...A_s) (B_1...B_s)}(k_a) := \frac{\partial^s (\mathrm{det}(T^{AB}(k_a)))}{\partial T^{A_1 B_1}(k_a) ... \partial T^{A_s B_s}(k_a)}.
\end{align}  
It can now be shown (cf. \cite{2018PhRvD..97h4036D}, \cite{2009JPhA...42U5402I}) that any adjunct matrix of order $s$ is subject to the following general form:
\begin{multline}
    Q_{(A_1...A_s) (B_1...B_s)}(k_a) = \\ \epsilon^{\sigma_1...\sigma_s} \epsilon^{\tau_1...\tau_s} \chi_{(\sigma_1)A_1}(k_a) \cdot ... \cdot \chi_{(\sigma_s)A_s}(k_a) \cdot \chi_{(\tau_1)B_1}(k_a) \cdot ... \cdot \chi_{(\tau_s)B_s}(k_a) \cdot \mathcal{P}(k_a),
\end{multline}
where $\mathcal{P}(k_a)$ is a homogeneous, order $2r-4s$ polynomial in the covector components $k_A$. We call this function the \textit{\textbf{principal polynomial}} of the given EOM.
Hence, in the infinite frequency limit, for (\ref{waveAns}) to describe a physically non-trivial, wave-like solution to the EOM, it is necessary for the corresponding wave covector $k_a$ to be a root of the principal polynomial $\mathcal{P}(k_a)$. 
Thereby, the principal polynomial encodes the entire information of the propagation of wave-like solutions with infinite frequency to the EOM. In particular, it thus contains the information to which spacetime domains such waves extend and thus might \textit{\textbf{causally influence}} (see also \cite{2012arXiv1211.1914K}, \cite{seiler2009involution} and \cite{2011PhRvD..83d4047R}).

For the special case of the EOM being the Euler-Lagrange equations of a diffeomorphism invariant, second-order Lagrangian, from the PDE (\ref{DiffeoEqn}) we can deduce that for all possible wave covectors, the following $4$ independent coefficient functions are contained in the kernel\footnote{This observation is in close relation with the existence of $4$ primary constraints in the associated Hamiltonian formulation that can also be derived from (\ref{DiffeoEqn}) (cf. \cite{TobiMaster}).}  of $T^{AB}(k_a)$: 
\begin{align}
   \chi_{(n)A}(k_a) =  C_{An}^{Cm}v_Ck_m.
\end{align}
In addition to defining admissible wave covectors of non-trivial solutions, the principal polynomial also provides information about suitable \textit{\textbf{initial data hypersurfaces}} that can serve as starting point for the initial value formulation of the theory, provided such a formulation exists.
\begin{theorem}
If the Cauchy-Problem of a given PDE is well-posed in a region of $M$, then the principal polynomial necessarily restricts to a hyperbolic polynomial on $T_p^{\ast}M$ for every $p$ contained in that region. Furthermore, exactly those hypersurfaces that have at every point a conormal which is hyperbolic w.r.t. $\mathcal{P}$ are admissible initial data hypersurfaces, i.e., serve the purpose of specifying initial data.
\end{theorem}
\begin{proof}
The proof can be found in \cite{Hormander1977} and also in \cite{Ivrii_1974}.
\end{proof}

Summing up we see that the principal polynomial of any hyperbolic EOM defines in each cotangential space $T_p^{\ast}M$, by means of its vanishing set $V_p$, the set of admissible, infinite frequency wave covectors, and further provides by its hyperbolicity cone $C_p$, information about possible choices of initial data hypersurfaces for the EOM. 
The situation is illustrated by figure \ref{Poly}.
\begin{figure}
\begin{minipage}{0.5\textwidth}
\begin{center}
\begin{tikzpicture}[tdplot_main_coords]
  \coordinate (O) at (0,0,0);
  
  \node (A) at (1,1,0) {$\boldsymbol{V_p^{(1)}}$};
  \draw[->, thick] (A) to [out = 70, in =320] (1,1,1.25);
  \node (B) at (-1,-1,0.5) {$\boldsymbol{V_p^{(2)}}$};
  \draw[->, thick] (B) to [out = 120, in =240] (-1.5,-1,2);
  \node (C) at (0,0,5) {$\boldsymbol{C}_p$};
  \draw[->, thick] (C) to [out = 210, in =90] (-0.5,-0.5,3.5);

  \begin{scope}[rotate around x = 25]
  \coneback[surface1]{-4}{2}{-15}
  \conefront[surface1]{-4}{2}{-15}
  \coneback[surface1]{4}{2}{15}
  \conefront[surface1]{4}{2}{15}
  \end{scope}

  \begin{scope}[rotate around y = 20]
  \coneback[surface1]{-3}{3}{-10}
  \conefront[surface1]{-3}{3}{-10}
  \coneback[surface1]{3}{3}{10}
  \conefront[surface1]{3}{3}{10}
  \end{scope}
\end{tikzpicture}
\end{center}
\end{minipage}
\begin{minipage}{0.5\textwidth}
\begin{center}
\begin{tikzpicture}[tdplot_main_coords]
  \coordinate (O) at (0,0,0);
  
  \node (A2) at (1,1,0) {$\boldsymbol{\widetilde{V}_p^{(1)}}$};
  \draw[->, thick] (A2) to [out = 70, in =320] (1,1,1.25);
  \node (B2) at (-1.5,-1.5,1) {$\boldsymbol{\widetilde{V}_p^{(2)}}$};
  \draw[->, thick] (B2) to [out = 120, in =180] (-1,-1,3);
  \node (C2) at (0.5,0.5,5.5) {$\boldsymbol{\widetilde{C}_p}$};
  \draw[->, thick] (C2) to [out = 210, in =90] (0,0,4);

  \begin{scope}[rotate around x = 10, rotate around y = 10]
  \coneback[surface1]{-4}{2}{-15}
  \conefront[surface1]{-4}{2}{-15}
  \coneback[surface1]{4}{2}{15}
  \conefront[surface1]{4}{2}{15}

  \coneback[surface1]{-3}{3}{-10}
  \conefront[surface1]{-3}{3}{-10}
  \coneback[surface1]{3}{3}{10}
  \conefront[surface1]{3}{3}{10}
\end{scope}
\end{tikzpicture} 
\end{center}
\end{minipage}
    \caption{Vanishing $V_p$, $\widetilde{V}_p$ and hyperbolicity cones $C_p$, $\widetilde{C}_p$ of two degree $4$ hyperbolic polynomials; in both cases the vanishing sets are given as union of the individual cones $V_p = V_p^{(1)} \cup V_p^{(2)}$, $\widetilde{V} = \widetilde{V}_p^{(1)} \cup \widetilde{V}_p^{(2)}$, the hyperbolicity are provided by their intersection $Cp = V_p^{(1)} \cap C_p^{(2)}$,  $\widetilde{C}_p = \widetilde{C}_p^{(1)} \cap \widetilde{C}_p^{(2)}$.}
    \label{Poly}
\end{figure}

\section{Perturbation Theory}
\subsection{power series ansatz}
1 page
\subsection{expansion of eeq}
1page
\subsection{expansion of polynomial}
2 pages

\section{Applications}
\subsection{computer algebra?}
1 page \cite{sparse-tensor}
\subsection{general relativity to second order}
1 page
\subsection{area metric gravity to second order}
2 page

\section{Conclusions}
2 pages


\printbibliography[
heading=bibintoc,
title={Bibliography}
]

\end{document}
